\section{以太网及 DoIP 配置流程}

基于 Vector Davinci 软件完成基本的以太网通信及 DoIP 诊断功能,可按照以下步骤进行配置:

\begin{enumerate}
    \item 添加 \lstinline{ctcddethswc} 确保原有工程添加以太网 SWC 后可正常调度,SWC 仅保留 Switch 硬件初始化的代码,其他不要添加。
    \item 添加以太网相关模块,包括协议栈基础模块及其他关联模块。处理所有错误后,仅添加以太网 Rx Tx 中断和以太网通道 comm \lstinline{mainfunction} 映射,确保工程仍正常运行。
    \item 检查 vlinkgen 模块是否有 \lstinline{.EthRam} var section 配置, 如果开启 cache,需要检查是否进行 cache 保护,及 SMPU flash 块的寄存器配置。
    \item 添加以太网初始化函数(BSW), ECUM \lstinline{memoryInit} 初始化函数,确保工程仍正常运行。
    \item 检查 Switch 硬件相关管脚设置,包括 PWMG 模块与 \lstinline{ctcddethswc},确保初始化管脚时序正确。
    \item 对 Switch 初始化函数进行背景任务映射,确保 Switch 可以正常初始化。
    \item 添加 eth 相关模块的 \lstinline{ mainfunction} 映射,修改 BSWM eth comm 通道控制,修改 SWC 以太网通信控制逻辑代码,验证 DoIP功能是否正常(初期需要默认使能激活线,否则无法建立 TCP 连接)。
\end{enumerate} 

接下来的各节分别针对各步骤中的关键配置进行梳理。

\subsection{SWC 基本结构及配置}
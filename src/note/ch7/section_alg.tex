\section{前导零与后导零计算}

前导零:即一个数的二进制,从 MSB 开始数,直到遇到第一个1,前面几有个零。如 \lstinline{0x00FFFFFF} 前导零为8; \lstinline{0xF0000000} 前导零为0; \lstinline{0x08010002} 前导零为4。

后导零则与之相反,从 LSB 开始计算。两种情况下,具体的代码实现见后两节。

\subsection{前导零}

此代码从 UCOS 实现源码中截取,详见博客\cite{65TiaoXiaoXiJiSuanQianDaoLingCYuYanShiXianU013073067DeBoKeCSDNBoKe}。
\lstinputlisting[
    style       =   C,
    language = C,
    caption     =   {前导零计算},
]{ch7/lead_zero.c}

\subsection{后导零}

\lstinputlisting[
    style       =   C,
    language = C,
    caption     =   {后导零计算},
]{ch7/back_zero.c}

\section{二进制数位翻转}

\begin{lstlisting}[language=C,style=C]
uint16 eth_bitReverse(uint16 value)
{
    value = (((value & 0xaaaa) >> 1) | ((value & 0x5555) << 1));
    value = (((value & 0xcccc) >> 2) | ((value & 0x3333) << 2));
    value = (((value & 0xf0f0) >> 4) | ((value & 0x0f0f) << 4));
    value = (((value & 0xff00) >> 8) | ((value & 0x00ff) << 8));

    return(value);
}
\end{lstlisting}

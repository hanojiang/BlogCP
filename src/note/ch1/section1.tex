\section{Port}

\subsection{数据类型定义}

\subsubsection{\lstinline{Port_aPinConfigDefault}}

已使用管脚的初始配置如下:

\begin{lstlisting}[language=C,style=C]
    typedef struct
    {
        VAR(Port_InternalPinIdType, PORT_VAR) u16SIUPin; /**< 使用PIN pcr number @brief Pin Defined on Part SIUx */
        VAR(uint32,                 PORT_VAR) u32MSCR;   /**< 寄存器 MSCR 初值    @brief Pad Control Register */
    #if defined(IPV_SIUL2_ODC_SUPPORT_U8)
        VAR(uint32,                 PORT_VAR) u32ODC;    /**<  not used          @brief Pad Output Driver Control Configuration */
    #endif /* defined(IPV_SIUL2_ODC_SUPPORT_U8) */
        VAR(uint8,                  PORT_VAR) u8PDO;     /**<  IO 初值电平 0 low 1 high 2 not care           @brief Pad Data Output */
        VAR(uint8,                  PORT_VAR) u8PDDir;   /**<  IO方向   0 in 1 out  2 inout             @brief Pad Data Direction */
        VAR(boolean,                PORT_VAR) bGPIO;     /**<  gpio 初始模式                  @brief GPIO initial mode*/
        VAR(boolean,                PORT_VAR) bDC;       /**<  方向是否可变,不可用                  @brief Direction changebility*/
        VAR(boolean,                PORT_VAR) bMC;       /**<  模式是否可变                  @brief Mode changebility*/
    } Port_Siul2_PinConfigType;

    static CONST(Port_Siul2_PinConfigType, PORT_CONST) Port_aPinConfigDefault[PORT_MAX_CONFIGURED_PADS_U16]=
{
    {(Port_InternalPinIdType)137, (uint32)0x00400000, (uint8)2, (uint8)0, (boolean)FALSE, (boolean)FALSE, (boolean)FALSE}
}
\end{lstlisting}

\subsubsection{\lstinline{Port_au16PinDescription}}
各管脚支持的复用模式汇总如下,每个管脚是否支持某一种复用功能,都可以通过算法算出。

\begin{lstlisting}[language=C,style=C]
    /* 每一行包括一种复用模式,263 个管脚每16个一组,分布在17个组中 */
    CONST (uint16, PORT_CONST) Port_au16PinDescription[24][17] = {};
\end{lstlisting}

支持的24中复用模式为:
\begin{lstlisting}[language=C,style=C]
    /* 所有管脚可用的模式都隶属于24种复用模式的其中一种,详见 Port_Cfg.h 文件中*/
/** @brief Port GPIO Mode */
#define PORT_GPIO_MODE                  ((Port_PinModeType)0)
/** @brief Port Alternate 1 Mode */
#define PORT_ALT1_FUNC_MODE             ((Port_PinModeType)1)
/** @brief Port Alternate 2 Mode */
#define PORT_ALT2_FUNC_MODE             ((Port_PinModeType)2)
/** @brief Port Alternate 3 Mode */
#define PORT_ALT3_FUNC_MODE             ((Port_PinModeType)3)
/** @brief Port Alternate 4 Mode */
#define PORT_ALT4_FUNC_MODE             ((Port_PinModeType)4)
/** @brief Port Alternate 5 Mode */
#define PORT_ALT5_FUNC_MODE             ((Port_PinModeType)5)
/** @brief Port Alternate 6 Mode */
#define PORT_ALT6_FUNC_MODE             ((Port_PinModeType)6)
/** @brief Port Alternate 7 Mode */
#define PORT_ALT7_FUNC_MODE             ((Port_PinModeType)7)
/** @brief Port Output Only Mode */
#define PORT_ONLY_OUTPUT_MODE           ((Port_PinModeType)8)
/** @brief Port Input Only Mode */
#define PORT_ONLY_INPUT_MODE            ((Port_PinModeType)9)
/** @brief Port Analog input Mode */
#define PORT_ANALOG_INPUT_MODE          ((Port_PinModeType)10)
/** @brief Port Input 1 Mode*/
#define PORT_INPUT1_MODE                ((Port_PinModeType)11)
/** @brief Port Input 2 Mode*/
#define PORT_INPUT2_MODE                ((Port_PinModeType)12)
/** @brief Port Input 3 Mode*/
#define PORT_INPUT3_MODE                ((Port_PinModeType)13)
/** @brief Port Input 4 Mode*/
#define PORT_INPUT4_MODE                ((Port_PinModeType)14)
/** @brief Port Input 5 Mode*/
#define PORT_INPUT5_MODE                ((Port_PinModeType)15)
/** @brief Port Input 6 Mode*/
#define PORT_INPUT6_MODE                ((Port_PinModeType)16)
/** @brief Port Input/Output 1 Mode */
#define PORT_INOUT1_MODE                ((Port_PinModeType)17)
/** @brief Port Input/Output 2 Mode */
#define PORT_INOUT2_MODE                ((Port_PinModeType)18)
/** @brief Port Input/Output 3 Mode */
#define PORT_INOUT3_MODE                ((Port_PinModeType)19)
/** @brief Port Input/Output 4 Mode */
#define PORT_INOUT4_MODE                ((Port_PinModeType)20)
/** @brief Port Input/Output 5 Mode */
#define PORT_INOUT5_MODE                ((Port_PinModeType)21)
/** @brief Port Input/Output 6 Mode */
#define PORT_INOUT6_MODE                ((Port_PinModeType)22)
/** @brief Port Input/Output 7 Mode */
#define PORT_INOUT7_MODE                ((Port_PinModeType)23)
\end{lstlisting}

\subsubsection{\lstinline{Port_aPadFunctInMuxSettings}}
该变量保存了各个管脚支持的输入复用模式,及其 \lstinline{IMCR[SSS]}值。需要注意的是,INMUX reg 的 offset 是512。所以真实的 reg 是结构体中的值加上 512。

\begin{lstlisting}[language=C,style=C]
CONST(Port_InMuxSettingType,PORT_CONST) Port_aPadFunctInMuxSettings[415]=
{

  /* INMUX settings for pad not available:  */
  { NO_INPUTMUX_U16, 0U},
  /* INMUX settings for pad PORT0:      {INMUX reg, PADSEL val} reg offset = 512*/
  /* EMIOS0_E0UC_0_X_IN input func */
  {0U, 2U}, 
  /* EMIOS0_E0UC_13_H_IN input func */
  {13U, 2U}, 
  /* FlexCAN_1_RX input func */
  {189U, 1U}, 
  /* INMUX settings for pad PORT1:      {INMUX reg, PADSEL val} */
  /* EMIOS0_E0UC_1_G_IN input func */
  {1U, 2U}, 
  /* FlexCAN_3_RX input func */
  {191U, 1U}, 
  /* INMUX settings for pad PORT2:      {INMUX reg, PADSEL val} */
  /* EMIOS0_E0UC_2_G_IN input func */
  {2U, 2U}, 
}
\end{lstlisting}

\subsubsection{\lstinline{Port_au16PadFunctInMuxIndex}}

每一个元素为一个输入管脚在 \lstinline{Port_aPadFunctInMuxSettings} 中的索引,但该索引不是直接对应的,而需要通过一定便宜运算:

\begin{lstlisting}[language=C,style=C]
#define PORT_INPUT1_MODE                ((Port_PinModeType)11)

index_in_Port_aPadFunctInMuxSettings = (index_in_Port_au16PadFunctInMuxIndex + PinMode) - PORT_INPUT1_MODE;
\end{lstlisting}

如 \lstinline{pad0} 在 \lstinline{Port_au16PadFunctInMuxIndex}中 \lstinline{index} 为 1,则对应 \lstinline{Port_aPadFunctInMuxSettings} 的第 \lstinline{(1 + PinMode) - 11}项。

该变量即明确的将当前所有管脚使用的复用功能从 \lstinline{Port_aPadFunctInMuxSettings} 选择一种,进行唯一确定。
\begin{lstlisting}[language=C,style=C]
CONST(uint16,PORT_CONST) Port_au16PadFunctInMuxIndex[264]=
{

  /* Index to address the input settings for pad 0*/
  (uint16)1, 
  /* Index to address the input settings for pad 1*/
  (uint16)4, 
  /* Index to address the input settings for pad 2*/
  (uint16)6, 
  /* Index to address the input settings for pad 3*/
  (uint16)7, 

 } 
\end{lstlisting}

\subsubsection{\lstinline{Port_aPadFunctInoutMuxSettings}}
该变量保存了各个管脚支持的输入输出复用模式。

\begin{lstlisting}[language=C,style=C]
CONST(Port_InoutSettingType,PORT_CONST) Port_aPadFunctInoutMuxSettings[PORT_INOUT_TABLE_NUM_ENTRIES_U16] = 
{

  /* Inout settings for pad PORT0:      {MSCR pcr number, MODE 复用模式, INMUX reg IMCR REG offset 512, PADSEL val} */
  /* EMIOS0_E0UC_0_X_IN_OUT input func */
  {0U, 17U, 0U, 2U}, 
  /* EMIOS0_E0UC_13_H_IN_OUT input func */
  {0U, 19U, 13U, 2U}, 
  /* Inout settings for pad PORT1:      {MSCR, MODE, INMUX reg, PADSEL val} */
  /* EMIOS0_E0UC_1_G_IN_OUT input func */
  {1U, 17U, 1U, 2U}, 
  /* Inout settings for pad PORT2:      {MSCR, MODE, INMUX reg, PADSEL val} */
  /* EMIOS0_E0UC_2_G_IN_OUT input func */
  {2U, 17U, 2U, 2U}, 
  /* Inout settings for pad PORT3:      {MSCR, MODE, INMUX reg, PADSEL val} */
  /* EMIOS0_E0UC_3_G_IN_OUT input func */
}

\end{lstlisting}

\subsubsection{\lstinline{Port_aPadSelConfigDefault}}
该变量实际存储了复用管脚的 \lstinline{IMCR} 寄存器的配置值,非 \lstinline{0xFFFFFFFFU} 值代表含有复用功能。并在 \lstinline{Port_Siul2_Init}中进行初始化。

\begin{lstlisting}[language=C,style=C]
static CONST(Port_Siul2_PadSelConfigType, PORT_CONST) Port_aPadSelConfigDefault[PORT_NMBR_INMUX_REGS_U16] = {};
\end{lstlisting}